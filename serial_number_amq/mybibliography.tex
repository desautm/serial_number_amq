\bigskip

\bibliographystyle{alpha}
\begin{thebibliography}{10}

\bibitem{Johnson} Roger W. Johnson (1994). Estimating the Size of a Population, {\em Teaching Statistics, 16}, (no. 2), pages 50-52.

\bibitem{Goodman1952} Leo A. Goodman (1952). Serial Number Analysis, {\em Journal of the American Statistical Association, 47}, (no. 260), pages 622-634.

\bibitem{Goodman1954} Leo A. Goodman (1954). Some Practical Techniques in Serial Number Analysis, {\em Journal of the American Statistical Association, 49}, (no. 265), pages 97-112.

\bibitem{Roberts1957} Harry V. Roberts (1957). Informative Stopping Rules and Inferences about Population Size, {\em Journal of the American Statistical Association, 62}, (no. 319), pages 763-775.

\bibitem{Volz2008} Arthur G. Volz (2008). A Soviet Estimate of German Tank Production, {\em The Journal of Slavic Military Studies, 21}, (no. 3), pages 588-590.

\bibitem{Ruggles1947} Richard Ruggles and Henry Brodie (1947). An Empirical Approach to Economic Intelligence in World War II, {\em Journal of the American Statistical Association, 42}, (no. 237), pages 72-91.

\bibitem{commodore64} Pagetable.com (4 février 2011). {\em How many Commodore 64 were really sold?}. Récupéré le 17 octobre 2018:  \href{https://web.archive.org/web/20160306232450/http://www.pagetable.com/?p=547}{https://web.archive.org/web/20160306232450/http://www.pagetable.com/?p=547}

\bibitem{iphone} Charles Artur (8 octobre 2008). {\em Why iPhones are just like German tanks.}. Récupéré le 17 octobre 2018:  \href{https://www.theguardian.com/technology/blog/2008/oct/08/iphone.apple}{https://www.theguardian.com/technology/blog/2008/oct/08/iphone.apple}

\end{thebibliography}
