%% Vous pouvez ajouter ici toutes les commandes que vous avez définies.
%% Par exemple

% \newcommand{\lr}[1]{\left(#1\right)}
% \newcommand{\abs}[1]{\left\vert#1\right\vert}

\DeclareMathOperator{\Arcsin}{Arcsin}
\DeclareMathOperator{\Arccos}{Arccos}
\DeclareMathOperator{\Arcsec}{Arcsec}
\DeclareMathOperator{\Arccsc}{Arccsc}
\DeclareMathOperator{\Arccot}{Arccot}
\DeclareMathOperator{\Arctan}{Arctan}
\DeclareMathOperator{\Arctanh}{Arctanh}
\DeclareMathOperator{\sech}{sech}
\DeclareMathOperator{\esp}{\mathbb{E}}
\DeclareMathOperator{\var}{Var}
\DeclareMathOperator{\cov}{Cov}
\DeclareMathOperator{\unif}{U}
\newcommand{\lr}[1]{\left(#1\right)}
\newcommand{\abs}[1]{\left\vert#1\right\vert}
\newcommand{\norm}[1]{\left\Vert#1\right\Vert}
\newcommand{\set}[1]{\left\{#1\right\}}
\newcommand{\crochet}[1]{\left[#1\right]}
\newcommand{\comeq}[1]{\quad\lr{\text{#1}}}

